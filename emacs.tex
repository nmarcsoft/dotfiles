% Created 2025-02-04 mar. 11:33
% Intended LaTeX compiler: pdflatex
\documentclass[11pt]{article}
\usepackage[utf8]{inputenc}
\usepackage[T1]{fontenc}
\usepackage{graphicx}
\usepackage{longtable}
\usepackage{wrapfig}
\usepackage{rotating}
\usepackage[normalem]{ulem}
\usepackage{amsmath}
\usepackage{amssymb}
\usepackage{capt-of}
\usepackage{hyperref}
\date{\today}
\title{Emacs notes}
\hypersetup{
 pdfauthor={},
 pdftitle={Emacs notes},
 pdfkeywords={},
 pdfsubject={},
 pdfcreator={Emacs 29.4 (Org mode 9.7.19)}, 
 pdflang={English}}
\begin{document}

\maketitle
\tableofcontents

\section{Basics}
\label{sec:org7f7682e}
To restart the deamon :
\begin{verbatim}
systemctl restart --user emacs
\end{verbatim}
\subsection{File managment and project :}
\label{sec:org090f311}
SPC + f : File managment
SPC + p : project managment

SPC + . : Find file

We can work on the context of a project using projectile (pre-installed in doom)
M - x : projectile-discover-project-in-directory : Permet d'automatiquement identifié les projects (recursivement depuis une root)
ajouter un .projectile à la racine d'un directory permet de le definir comme un project

SPC + p + p : permet de lister tous les projects.
\subsection{Buffers :}
\label{sec:org035acf5}
\begin{quote}
Les plus utiles
\end{quote}
SPC + b : pour intéragir avec les buffers
SPC + b + b : changer de buffer simplement en fonction du nom
SPC + b + i ; intéraction avec les buffers (Del)

SPC + b + i ; intéraction avec les buffers (Del)
\section{Agenda}
\label{sec:orgc411d42}
Il faut spécifier dans quel directory notre org-agenda va regarder. Pour ma part : \textasciitilde{}/Documents/org/
Ensuite, voici comment rajouter des tâches :

\#+begin\textsubscript{quote}
\subsubsection{{\bfseries\sffamily TODO} Complete this task eventually}
\label{sec:org604dbdf}

\subsubsection{{\bfseries\sffamily TODO} Complete this task by September 30th}
\label{sec:org0f6b3be}
\subsubsection{{\bfseries\sffamily TODO} Start this task on September 30th.}
\label{sec:org336d922}
\#+end\textsubscript{quote}
\section{Utils}
\label{sec:org5165c13}
\subsection{Notes :}
\label{sec:org6e6bfe3}
En org-mode, on peut utiliser C-c C-e pour pouvoir convertir notre fichier ne divers format.
On peut donc facilement avoir une preview de nos documents
\end{document}
